The objectives of this thesis stem from the farmers' need to derive data-based farming decisions from data measured of their fields. While aggregated field-level data provides general guidelines, the actions and interventions are performed at the intra-field scale. The decisions also have to be made within an actionable time frame during the growing season. However, the data alone is not enough. As unmanned aerial system (UAS) overflights can be utilized to provide frequent image snapshots of distinct states of fields and crop growth, predicting an outcome from these data is a difficult task humans. What is needed is an automated decision engine based on data-based machine learning techniques, capable of performing intra-field predictions using current state of crop development. Furthermore, this decision engine should be integrated into a holistic farming decision support system (DSS) to fully utilize the capabilities of modern sensors, connectivity and automatic data processing. This enables the farmers to make more informed decisions on what actions to take and in which parts of distinct fields. 

In this chapter we will first review the relevant background and the current state-of-the-art smart farming and data sources in the context of crop yield prediction. While smart farming encompasses a broader farming context, from soil and water management to utilizing modern technology to optimize farming processes, we will constrain the discussion to the context of crop field management and crop yield estimation. 

The chapter is constructed as follows. In the first section the current studies of data-driven smart farming are reviewed. This is to gain a proper view of the application context for machine learning models, which are discussed in Chapter \ref{ch:dl-in-agriculture}. After that, data from distinct sources and the use thereof in agriculture-related modeling tasks are reviewed. Remote sensing is of particular interest, as it has been an active research area for a several years already. Other data sources, such as soil and weather data, are also discussed. In addition to reviewing relevant studies, we will also describe the data utilized in the studies related to this dissertation. In the last section of this chapter the modelling task of crop yield prediction is reviewed.


\section{Precision agriculture and smart farming}
\label{sec:smart-farming-review}

The technologization towards the modern age farm has been a steady process, ongoing for several centuries. The first steps in this process were taken during the 18th century with important gradual developments in crop rotation and selective breeding techniques. After the World Wars, the farms were quickly mechanized and farming processes started to get more industrialized. Manual labor and animal work force wre replaced with more effective machinery. As digital computation resources became more common via mainframe architectures starting at late 1960s, software products were adopted as common tools for agronomical counselling instutions and, thus, farming management practices. The introduction of the internet and developments in telecommunication, sensor and computer technologies enabled the farms gain increasingly detailed grasp of different areas of crop farming. The introduction of digital computation first transformed the data handling and computation processes of agricultural experts and advisors, starting with punch hole cards and progressing towards software applications \cite{Syvajarvi2016}. 

The developments in sensors, information technology (IT) systems and the general adoption of digital farm management and decision support systems have further driven the transformation towards what is known as precision agriculture. Precision agriculture is seen to encompass location-based technologies, processes and management concepts to better account for the intra-field variability for increased gains. While precision agriculture is focused mainly on farming operations in the field, smart farming extends the combination of physical sensors, IT systems and low latency connectivity to a holistic and automated farm management framework. This view is shared between multiple studies. Sundmaeker et al. \cite{Sundmaeker2016} position precision agriculture within smart farming as do Wolfert et al. \cite{Wolfert2017d} and Tantalaki et al. \cite{Tantalaki2019}. While Rose and Chilvers use the terms more interchangeably, their use of the term smart farming implies a larger framework, encompassing precision agriculture as technology and sensor oriented subarea \cite{Rose2018}.

Being conceptual frameworks, both precision agriculture and smart farming have experienced developments via advancements in distinct technological areas. This is reflected in recent studies. As discussed by Klerkx et al. in their review of digital agriculture, technologies such as precision farming, internet of things (IoT), machine learning (ML), deep learning (DL) and robotics have been a focus in an increasing number of agriculture related studies \cite{Klerkx2019}. In a recent review of machine learning (ML) based crop yield prediction, Van Klompenburg et al. observed an increase in publications utilizing novel data-based modelling concepts starting from 2013 \cite{VanKlompenburg2020}. Similar observation is made in a review of the use of deep learning (see Chapter~\ref{ch:dl-in-agriculture}) in agriculture by Tantalaki et al. \cite{Tantalaki2019}. They observe a monotonic increase of 249 \% on the average the number of annually published agriculture related deep learning focused studies between 2016 and 2019. 


\subsection{Decision support systems for agriculture}
\label{subsec:dss-for-agri-review}

The concepts of smart farming and digitalized agriculture are among the most relevant topics in the agricultural research domain. The key elements in smart farming revolve around data collection and utilization \cite{Klerkx2019}, data-based decision making \cite{Kamilaris2017}, interconnectivity of cyber-physical systems \cite{Zamora-Izquierdo2019}, automation of farming processes \cite{Zamora-Izquierdo2019} and improved management of farm processes \cite{Tantalaki2019}. 

One of the core elements of smart farming is data collection. Small and interconnected sensors, being more generally labelled as IoT-sensors, are utilized in tandem with sensors installed on farming equipment and machinery to produce a multi-source data stream from the farm. Data accumulated over time paints a holistic picture of the farm and its operations. Novel AI-related techniques further facilitates data-based decision making via insight extraction and estimation. This enables the farmers to base their decisions on measured data in timely and accurate manner \cite{Sundmaeker2016}. Moreover, the developments in soil sensors planted in crop fields enable the farmers to remotely monitor their fields, which in turn allows them to make more informed decisions on actions to take \cite{Tantalaki2019}. Being a subject closely related to the IoT, performing data aggregation and analysis on-site via edge computing is another projected direction for agricultural cyber-physical systems \cite{Zamora-Izquierdo2019}. 

Sensors, data and insights require effective management systems. A holisting agricultural management system addresses a farm's needs on multiple levels, such as accounting, traceability and on-farm process management. The management systems are also required to connect the farm to its stakeholders, such as consumers, public authorities and actors in the food value chain \cite{Tantalaki2019}. With the developments of the IT-sector in general, farm management solutions have also shifted from locally installed software to cloud-based services \cite{Zamora-Izquierdo2019}. This change further opens up new possibilities for data-based decision making \cite{Rose2018}. Especially, resource-intensive modelling techniques are easier to employ with dedicated servers. The adoption of smart farming practices makes the farm effectively a producer and manager of goods and operations related data. Being part of a larger agricultural ecosystem, the data generated on-farm is seen to benefit other instances, such as actors in the logistic chain and counselling institutions \cite{Kamilaris2017}.

When smart farming is viewed as a holistic operating framework, equipment and independent systems add formidably to the complexity of the whole. There is a true need to further develop the integration of sensors, equipment, monitoring and management systems \cite{Sundmaeker2016}. This calls for cooperation of business actors operating in the domain of smart farming, with IT operations being a focus of the development due to integrations. With working integrations, the benefits of accurate and timely automation can be reaped \cite{Zamora-Izquierdo2019}.

Several commercial decision support systems exist in the domain of agriculture. As the products are generally suites of modular and specialized applications, we will review the products only generally. Minun Maatilani (Mtech Digital Solutions Oy, Vantaa, Finland) provides the farmers with web-based applications for cattle and crop farm operations planning, accounting and management. There are explicit modules available for smart farming, which include features for managing cropping plans, creating and exporting fertilization tasks for machinery, importing of UAV data and yield maps. Satellite data is utilized to provide timely views of fields. Next Farming (FarmFacts Gmbh, Pfarrkirchen, Germany) has applications for crop and fertilization planning, fleet management, creation and management of prescription tasks for machinery. Users can import information about their fields, such as biomass, soil and yield maps. The software suite includes smart farming services such as UAV management, seeding and fertilication optimization and supplying geographic information system (GIS) data. 365FarmNet (365FarmNet Gmbh, Berlin, Germany) contains applications for farm management, crop cultivation and herd management. Via partner applications the suite provides the users with satellite-based field monitoring, crop, seed and fertilizer planning and fertilization optimization. MyEasyFarm (MyEasyFarm, Bezannes, France) contains applications for plant and plot management, task management, imported data analysis (soil, yield, etc.) and task monitoring.


\subsection{Crop yield prediction}
\label{subsec:crop-yield-prediction-review}

Crop yield prediction, the primary focus of this study, is deemed one of the most challenging problems in the realm of smart farming, the latter encompassing a large variety of sub-tasks and smaller goals. Predictive yield modelling is seen to help farmers pinpoint problem areas in their fields \cite{Shidnal2019}, guide management decisions and reduce business risk \cite{Filippi2019} and provide vital information for the food supply chain \cite{Zhao2020}. As discussed by Triantafyllou et al. \cite{Triantafyllou2019}, crop and plant yield estimation is crucial when the goal is to optimize field-wise yields in cost-effective and proactive manner. In their study of a holistic remote sensing system architecture, predictive models are positioned adjacent to data analysis, information management and data processing modules within what they call "managament layer". Management layer provides managament logic to the applications operated by the users, farmers or agricultural experts. 

According to {\"{U}}nal in their review of deep learning method utilization in the context of smart farming, yield estimation is one of the most common agriculture related keywords present in the reviewes 120 studies \cite{Unal2020}. The output, the harvested crop yield, is affected by a variety of environmental, crop-related and farmer-induced factors. Data-based modelling techniques, namely deep learning models, excel with such multivariate and non-linear data \cite{Xu2019}. In their review of machine learning based crop yield prediction, van Klompenburg et al. \cite{VanKlompenburg2020} observe that the data sources often present in crop yield prediction studies include soil and crop information, climatological data, information about the nutrients and actions taken by the farmer. 

In addition to gathering data from multiple sources, it is also necessary to collect data across multiple years. As Filippi et al. discuss, having the data cover larger time spans (\emph{temporal coverage}) is deemed more important than having the field related data span larger areas (\emph{spatial coverage}) \cite{Filippi2019}. A key aspect to using crop yield prediction in a smart farming DSS is to enable the farmer to decide on actionable items. Predicting the intra-field variability allows identifying underfperfoming areas in the fields \cite{Tantalaki2019}. With the increase of spatial resolution in predictions, the goals of precision agriculture are also easier to attain by focusing on distinct problem areas instead of treating the whole field in uniform manner.


\section{Data sources}
\label{sec:data-sources-review}

Remote sensing has played a significant role in advancing crop field monitoring during the past decades and is considered one of the most important technolgies for precision agriculture and smart farming \cite{Tsouros2019}. According to Khanal et al., the publicly accessible high-altitude satellite systems, such as Sentinel (ESA, Paris, France) and Landsat (USGS, Reston, Virginia, USA), have been a major catalyst in propelling remote sensing based agricultural research forward \cite{Khanal2020}. Other key factor in this progression has been the developments in computation and storage capabilities of such data. While high altitude monitoring is good for observing larger areas, low-altitude unmanned aerial vehicles (UAV) and unmanned aerial systems (UAS) are used to capture information in greater detail. According to {\"{U}}nal et al. in their review of deep learning in smart farming, the use of UAVs in recent agricultural deep learning studies is so prevalent that their use can be considered an integral part of the smart farming framework \cite{Unal2020}.  

Agricultural data is known to be heterogeneous \cite{Kamilaris2017}. According to Wolfert et al., this stems from the heterogeneity of the means of data accumulation, which include various remote sensing platforms, ground-based sensors and human-inputted data \cite{Wolfert2017d}. Another source of data heterogeneity are the objects of data measurement, i.e. the environment, machinery and operational records. In a recent review of the use of multisource and multitemporal data in remote sensing, Ghamisi et al. conclude that the increased availability of data from multiple sources accompanied with advances in computational tools has a positive effect on data-based modelling, increasing the efficiency and performance of the models \cite{Ghamisi2019}. Their review focuses solely on studies utilizing high and low altitude remote sensing platforms and their sensors. The sensor types include those of visible light RGB, multispectral, hyperspectral and laser imaging, detection and ranging, hereafter called lidar as per \cite{Ghamisi2019}. In a review of big data practices in agriculture, Kamilaris et al. observe that multiple data-based modelling studies in the domain of agriculture also utilize data from other sources \cite{Kamilaris2017}. These sources include weather stations, geospatial data, soil sensors, historical datasets and records kept by organizations, institutions and governments. 


\subsection{Low-altitude unmanned aerial vehicles}
\label{subsec:data-uavs-review}

UAVs have been utilized for the past decade in multiple studies related to remote sensing, data-based modelling and agriculture. Recently published reviews show that the number of UAV-related studies has substantially grown. Therefore it is more beneficial to perform a metareview on recent reviews focused on low-altitude remote sensing and its applications.

To preface the review of UAV usage in the context of remote sensing and crop yield estimation in agriculture, it is necessary to note that UAVs utilized in studies are mainly just aerial platforms to which the sensors are mounted. This is in contrast to several commercially available UAVs with integrated RGB cameras. Generally, there are five types of sensors present in the recent studies: visual RGB, multispectral, hyperspectral, thermal and lidar sensors \cite{Tsouros2019,Xie2020,Messina2020}. As implied by the name, visual RGB sensors capture the red, green and blue bands of the visible light spectrum in the 400-700 nm wavelength range \cite{Xie2020}. Multispectral sensors usually add one to several additional channels from select wavelengths in the near-infrared (NIR) wavelength region of 780-2500 nm. Hyperspectral sensors are used to capture a continuous spectral range from visible to NIR wavelengths \cite{Xie2020}. Thermal sensors measure the infrared radiation in the 3-8 $\mu$m wavelength region \cite{Messina2020}. Compared to the sensors mentioned above, lidar is an active sensor, emitting the signal and measuring its reflection from various surfaces \cite{Xie2020,Khanal2020}. Visual RGB sensors are generally the easiest to operate and cheapest to acquire. Multispectral and hyperspectral sensors need often be acquired and mounted separately and they cost considerably more than RGB sensors. Thermal and lidar sensors are among the most expensive UAV-mountable sensors \cite{Tsouros2019}.

Khanal et al. reviewed accomplishments, limitations and opportunities of remote sensing in agriculture \cite{Khanal2020}. Searching for studies related to remote sensing and agriculture, they discovered 3679 studies during the 20-year period from 2000 to 2019. The number of UAV related studies, according to their research, started to increase after 2013. The annual numbers rose from a handful at beginning of the considered period to well over a hundred UAV related studies published in 2019. Focusing on recent and major references, they review the applications of remote sensing in precision agriculture. They observe that UAVs have been utilized in the following applications:

\begin{itemize}
    \item topographical mapping (1/3)
    \item tile drainage locationing (2/5)
    \item soil moisture and temperature mapping (3/8)
    \item crop emergence and density monitoring (5/5)
    \item nitrogen stress monitoring (1/3)
    \item crop disease monitoring (3/8)
    \item weed identification and classification (3/4)
    \item yield prediction (2/4).
\end{itemize}

The numbers after the items indicate the number of UAV related references reported out of all reported references for an application. Overall, they observe UAV related studies making up 16.3 \% of the studies considering remote sensing in agriculture during 2015-2019. The majority of the studies they reviewed focused on satellite sources. Recently, however, there has been an increase in studies utilizing UAV-based data to perform data analysis and data-based modelling with high resolution data. In the studies they selected for closer inspection, the UAVs were equipped with visual, multispectral and thermal sensors for various applications. In their view, UAV platforms provide a reasonable means to gather high-frequency and high-resolution remote sensing data with. Citing US prices, they report UAV data collection to cost approximately 9.9\$/ha. They also point out that operating UAVs is constrained by weather conditions, limited flight time and payload.

Touros et al. conducted a review on UAV-based applications for precision agriculture \cite{Tsouros2019}. They reviewed 100 research papers published between years 2017 and 2019. According to them, UAVs can be used to produce high to ultra-high resolution images of crop fields by varying the flying height. They observe that UAVs are utilized in the following applications:

\begin{itemize}
    \item crop growth monitoring (65.6 \% of studies)
    \item weed mapping (12.5 \% of studies)
    \item crop health monitoring (6.3 \% of studies)
    \item crop irrigation management (5.2 \% of studies).
\end{itemize}

While other applications were observed in addition to these, the four formed the majority (89.6 \%). Limited to these application contexts, four distinct categories of sensors were observed. These were multispectral (56.0 \%), RGB (33.6 \%), thermal (6.0 \%)  and hyperspectral (4.4 \%). They conclude that the use of various vegetation indices derived from multispectral data are the most effective in crop parameter monitoring. Overall, they observed more than 30 distinct crop species among the reviewed studies. For this dissertation, crop grotwh monitoring as an application context is of the greatest interest, while crop yield prediction is considered a part of it in the review. RGB and multispectral sensors are reported being the most utilized for this application. Machine learning methods were observed as being able to exploit data from all sensor types, both separately and conjoined. 

Xie and Yang reviewed the current state-of-the-art of UAV-mounted sensor utilization in plant phenotypic trait monitoring and estimation \cite{Xie2020}. Main phenotypic traits include plant yield, biomass, height, leaf area index, chlorophyll content and nitrogen content. Overall, they observed 18 different plant varieties as the targets for UAV-based sensing in their review. Concluding from plant yield estimation focused studies, they suggest using RGB and multispectral sensors with UAVs. Biomass, height and leaf area index are also treated as proxy variables for plant yield. Biomass estimation was performed mainly with RGB and multispectral sensor data. Lidar was observed as the dominant sensor type with canopy height estimation. Leaf area index was mostly estimated using various vegetation indices derived from multispectral data with some studies resorting to RGB sensors as well. In conclusion, they observe that RGB and multispectral sensors are used predominantly in plant related monitoring and estimation studies. This is attributed to lower sensor costs, sensor lightness and the ease of data collection and analysis. Multispectral data is seen, however, to be crucial for some crop related monitoring and modelling contexts where vegetation indices based on he infrared part of the spectrum are utilized.

Messina and Modica reviewed the current state of the art of UAV thermal imagery and its applications \cite{Messina2020}. Thermal sensors detecting the infrared radiation are used mainly to monitor ground surface temperature. It has been observed to be a rapid response variable to plant growth, yield estimation and stress factor evaluation. Compared to other sensor types, such as RGB and multispectral, operating thermal sensors requires more care. Environmental variables, such as humidity, clouds, dust and time of day, can impede the data acquisition process. Calibration of sensors and measuring environmental variables near the imaged objects is strongly suggested to perform corrections during data processing. The most utilized applications for UAV-mounted thermal sensors observed in their review were the following:

\begin{itemize}
    \item water stress detection and monitoring (23 studies)
    \item phenotyping (5 studies)
    \item yield estimation (4 studies).
\end{itemize}


\subsection{High-altitude satellite systems}
\label{subsec:data-satellites-review}

Remote sensing studies conducted with free and commercial satellite data have been common for longer than comparable studies with UAVs. For several years already, satellite data has been considered a core data source in the smart farming framework \cite{Wolfert2017d}. Some of the often utilized satellite systems with their specifications are given in Table~\ref{tab:satellites-review}, but it is to be noted that there exists much larger number of past and presently operational satellite missions. For reference, see the database of satellite missions at \cite{SatelliteMissionsDirectory}.

% Please add the following required packages to your document preamble:
\begin{table}[htb]
    \scriptsize
    \centering
    \caption{Some of the commonly referenced satellite systems present in remote sensing and agriculture related studies.}
    \label{tab:satellites-review}
    \begin{tabular}{@{}llllllll@{}}
    \toprule
    \textbf{Satellite}             & \textbf{\begin{tabular}[c]{@{}l@{}}Spatial \\ Resolution\\ {[m/px]}\end{tabular}} & \textbf{\begin{tabular}[c]{@{}l@{}}Revisit\\ Time\\ {[days]}\end{tabular}} & \textbf{\begin{tabular}[c]{@{}l@{}}Number of\\ Satellites\end{tabular}} & \textbf{\begin{tabular}[c]{@{}l@{}}Spectral\\ Channels\end{tabular}} & \textbf{\begin{tabular}[c]{@{}l@{}}Spectral\\ Range\\ {[$\mu$m]}\end{tabular}} & \textbf{\begin{tabular}[c]{@{}l@{}}Launch\\ Year\end{tabular}} & \textbf{\begin{tabular}[c]{@{}l@{}}Open\\ Access\end{tabular}} \\ \midrule
    Landsat-7 \cite{Landsat7}      & 15-60                                                                             & 16                                                                         & 1                                                                       & 8                                                                    & 0.441-12.36                                                                    & 1999                                                           & Yes                                                            \\
    Landsat-8 \cite{Landsat8}      & 15-60                                                                             & 16                                                                         & 1                                                                       & 11                                                                   & 0.435-12.51                                                                    & 2013                                                           & Yes                                                            \\
    Sentinel-2 \cite{Sentinel2}    & 10-60                                                                             & 5                                                                          & 2                                                                       & 13                                                                   & 0.426-2.377                                                                    & 2015                                                           & Yes                                                            \\
    WorldView 2 \cite{WorldView2}  & 0.31-1.84                                                                         & 1.1                                                                        & 1                                                                       & 9                                                                    & 0.450-2.365                                                                    & 2009                                                           & No                                                             \\
    WorldView 3 \cite{WorldView3}  & 0.31-1.24                                                                         & <1 to 4.5                                                                  & 1                                                                       & 29                                                                   & 0.450-2.365                                                                    & 2014                                                           & No                                                             \\
    PlanetScope \cite{PlanetScope} & 2.7-3.2                                                                           & 1                                                                          & 140                                                                     & 4                                                                    & 0.455-0.860                                                                    & 2016                                                           & No                                                             \\
    Gaofen 1 \cite{Gaofen1}        & 2-16                                                                              & 4                                                                          & 1                                                                       & 5                                                                    & 0.450-0.900                                                                    & 2013                                                           & Yes                                                            \\
    Gaofen 2 \cite{Gaofen2}        & 0.81-3.24                                                                         & 5-69                                                                       & 1                                                                       & 5                                                                    & 0.450-0.900                                                                    & 2014                                                           & No                                                             \\ \bottomrule
    \end{tabular}
\end{table}

Since the launches of higher resolution satellite systems, such as Landsat 8 in 2013 and Sentinel-2 in 2015, and the opening of their data, the usage of data from remote sensing satellites in various application domains has become more feasible. As Chivasa et al. discuss, a review of maize yield estimation applications based on remote sensing, coarse-resolution satellite data was largely unusable with smaller sized fields on the African continent \cite{Chivasa2017}. The values in a pixel corresponding to a field would effectively always be contaminated with data unrelated to the field. Furthermore, to estimate a yield produced by a spatially irregularly shaped field requires data at high enough resolution to constrain the field data within reasonable borders.

Khanal et al. calculate that 64 \% of the 3679 remote sensing and agriculture related studies published in and after the year 2000 utilize satellite-based data \cite{Khanal2020}. They also observe satellite data-based studies being more prevalent than UAV utilizing studies in the years from 2000 to 2010. According to their research focused on selected studies, satellite data have been utilized in the following agriculture related applications:

\begin{itemize}
    \item tile drainage locationing (1/5)
    \item soil moisture and temperature mapping (3/8)
    \item nitrogen stress monitoring (1/3)
    \item crop disease monitoring (1/8)
    \item weed identification and classification (1/4)
    \item yield prediction (1/4)
    \item grain quality assessment (1/3)
    \item crop residue assessment (3/4).
\end{itemize}

Numbers in brackets indicate the satellite data utilization counts in all papers related to the particular application context. The numbers suggest that satellite-based studies are in the minority when compared to UAV studies. This, however, might be attributable to the authors of the review as they seem to put more focus on high resolution studies. UAVs and mid-altitude manned aircrafts are better at producing high resolution data. Regarding economics, medium resolution satellite data is largely open access and free to use. High-resolution satellite data is reported to cost from 1.28 USD/km$^2$ (5 m/px resolution) to 25 USD/km$^2$ /0.5 m/px resolution). Compared to UAVs at 9.9 USD/ha, the price with commercial satellites is cheaper for larger areas. Smaller areas require economic evaluation case-by-case, as minimum order size is enforced with commercial high-resolution satellite data.

In another recent study, Karthikeyan et al. review remote sensing applications regarding crop growth, irrigation and crop losses \cite{Karthikeyan2020}. Focusing on the international and global scale, they assess uses of current operational satellite systems in performing large-scale data acquisition for monitoring and modelling of crop growth, losses and irrigation. While they affirm that data gathered on site with UAVs and sensors is more efficient in smaller scale, they view satellites as unrivalled in continuous monitoring of larger areas. Regarding crop growth, they observe that the multispectral and hyperspectral instuments in satellite platforms enable the use of various vegetation indices relevant to crop assessment. To effectively utilize vegetation indices, the utilized satellite systems are required to have at least mediocre spatial resolution. Similar to \cite{Chivasa2017}, they acknowledge the problem of pixel value contamination for agricultural use with too coarse resolutions. For irrigation monitoring, they observe visible, infrared and microwave sensors being utilized. Recently, data fusion has also been utilized in generating yearly irrigation maps dating decades to the past. In these studies satellite data were complemented with other data, including weather, soil and topographical information. While they assess several application contexts, they conclude that higher resolution is often needed.


\subsection{Weather data}
\label{subsec:data-weather-review}

Optical sensing is of crucial importance when performing spatial modelling in the context of crop yield prediction. While sensing crop growth stages is helpful, gathering data about the environment is mandatory to distinquish the effects of a crop type's phenological factors from external factors. In a study of a holistic remote sensing monitoring system, Triantafyllou et al. position weather data logging on par in terms of importance with other sensors installed and planted on site \cite{Triantafyllou2019}. Reported weather related environmental factors include wind speed and direction, atmospheric pressure, light intensity, solar radiation and rainfall. In addition to specifically installed sensors, nationally collected weather data and forecasts have also been used \cite{Kamilaris2017}.

Sun et al. conducted a multisource soybean yield prediction study at US county scale \cite{Sun2019}. In addition to remote sensing and yield data, they utilized historical daily weather data accumulated in the Google Earth Engine \cite{GoogleEarthEngine}. The weather data, namely precipitation and atmospheric pressure, were utilized as rasters with 1 km/px ground sample distance. Analyzing their results, they attribute some of the lowest soybean yields partially to extreme weather. However, they note that singling out the effects of external factors on yield is complex. Their conclusion is that weather data accompanied with remote sensing data form a sufficient data set with which to predict soybean yields using coarse resolutions.

In a study of maize growth stage prediction, Yue et al. utilized a county level meteorological data set as the predictor data in China \cite{Yue2020}. The weather data consisted of daily aggregates for humidity, atmospheric pressure, temperature, precipitation, wind speed and sunlight amount measured from a single weather station. The temporal range of the data is reported being from 1981 to 2017. The weather data were temporally aligned with the maize growth data to facilitate timely estimation of the maize growth stage from meterological data only. Using days of growth as the predicted value, they report average absolute error of 1.06 days.

Wolanin et al. utilized time series of remote sensing and weather obsevations to estimate crop yields in the Indian Wheat Belt \cite{Wolanin2020}. They utilized daily aggregates of temperature, precipitation, water vapour deficit, short-wave radiationa and day length information. In addition they utilized vegetation indices calculated from remote sensing data. They trained their models with data from multiple years to potentially observe the effects of environmental factors on the crop yield. Their conclusion is that while vegetation indices capture the effects of environment and render weather data somewhat redundant in their modelling approach, analysis of the model's utilization of meterological features provides insights into other study areas, such as crop breeding.


\subsection{Soil data}
\label{subsec:data-soil-review}

Being the base of the growth for crops, soil and its composition plays a major role in how the plants grow and produce grains. As Tantalaki et al. show in a review of novel data-based applications in precision agriculture, soil and its features are commonly the target of modelling \cite{Tantalaki2019}. However, studies have also been conducted where soil and ground related data are used as predictor values.

In a review of machine learning and crop yield prediction, van Klompenburg et al. observed soil type and soil maps being utilized often in recent data-based modelling studies in the context of agriculture \cite{VanKlompenburg2020}. Invidual spatial soil features included soil type, pH, cation exhange capacity and location. Soil related features were, overall, observed to be the most prevalent group of data features present in the reviewed studies. These feautures were observed as predictors of crop yield 54 times, while second most popular group, solar information, saw 39 uses as predictor values in similar setting. Soil information was also utilized both as predictor and predicted values in the reviewed studies. 

Filippi et al. collected a multisource data set to estimate crop yields \cite{Filippi2019}. Soil related features included soil electrical conductivity, potassium, uranium, thorium, clay and sand content. These acquired data were processed to a resolution of 10 m/px. Other data sources included remotely sensed vegetation indices as well as received and forecasted precipitation. Regarding the use of soil data, they conclude that soil maps and geophysical data were not as significant predictors as initially assumed. However, they observed correlations between soil and ground related predictor values and point out that this might actually mask their combined significance.

Khanal et al. utilize soil related features in their study of machine learning based intra-field corn yield and soil feature estimation \cite{Khanal2018}. Using a single field for their study, the soil was sampled at intervals of one acre or 0.40 ha. From the samplings, the ratios for soil organic matter, potassium and magnesium were extracted. Cation exhange capacity and pH were also measured. These features were, however, treated as target values. The inputs for estimation consisted of high resolution multispectral ($<$1 m) images and digital elevation model data. Inputs were spatially aligned with corresponding soil samples, forming the soil-related input-target data set. In their study the authors compared statistical, linear and non-linear models. Spatial models, such as CNNs, were not, however, taken into comparison.


\subsection{Lidar and topographical maps}
\label{subsec:data-topo-review}

As already mentioned as one of the sensors mountable to UAVs, lidar is often utilized when remotely sensed elevation information is acquired. As pointed out by Khanal et al., topographical features affect preseason farming management decisions, impacting field's water economy and soil quality \cite{Khanal2020}. Another common application context are tree and forest related studies \cite{Salmivaara2020}.

In a review of multisource and multitemporal remote sensing data fusion, Ghamisi et al. point out multiple studies in which lidar data has been utilized \cite{Ghamisi2019}. In raw form a lidar sensor produces multidimensional point cloud of data, which contains information about locations and altitudes of the points. They observe that lidar is oftern accompanied with a separate hyperspectral sensor. One of the main reasons for this is that, lidar generally lacks spectral information often necessary. This is true for scene classification, for example.

While lidar sensors provide exceptional accuracy when a digital elevation model (DEM) is required, other approaches exist for mapping the topography of the target area of interest. Recent advances in UAV-based photogrammetry, i.e. modelling structure from images taken from different angles, provide an alternative approach to map intra-field topographical variability \cite{Khanal2020}. Namely, the advances in UAV-based photogrammetry have enabled producing DEMs from considerably cheaper and lightweight RGB sensors. These methods, however, lack canopy penetration when compared to lidar \cite{Maimaitijiang2020}.


\subsection{Yield maps}
\label{subsec:data-yield-review}

Crop yield estimation is an important topic in the context of smart farming and precision agriculture. Correctly estimating crop yield mid-season enables the farmer to proactively focus on problem sectors of their fields. This can lead to increased profits via increased yields and cost savings due to the ability to focus on distinct areas instead of performing uniform treatments. Traditional approach to measuring the crop yield from a field consists mainly of weighing harvested grains and calculating an average for a field. To facilitate supporting intra-field decision making, combine harvesters can be equipped with yield monitoring systems. There exists various methods of measuring the harvested yield. These methods include optical measurement and kinetic mass flow sensors. Additionally, yield monitoring systems utilize global navigation satellite system (GNSS) to assign location information to the measurements. Accurate yield maps are necessary to model intra-field yield variability \cite{Khanal2020}.

As shown by van Klompenburg et al. in their review of 50 machine learning based crop yield prediction studies, performing yield estimation with input data from various sources is a current and developing research topic \cite{VanKlompenburg2020}. The use of spatial, i.e. geolocated yield information at the intra-field scale, crop yield data is becoming common. While the authors of the review do not examine the formats of used data w.r.t. spatially arranged yield targets, the notable presence of CNN architectures (36.4 \%) indicates presence of spatial input-target pairs as training samples. This is in contrast to crop yield estimation studies, where crop yield information is aggregated over larger areas, such as counties \cite{Lin2020, Wang2020, Sun2019}. 

Filippi et al. utilize 10 m/px resolution yield maps as target data in their study of crop yield estimation using multi-layered and multi-farm data with machine learning methods \cite{Filippi2019}. Yield information was initially generated by combine harvesters equipped with yield mapping sensors and was then processed to generate yield maps for the study. In addition to using yield maps as targets, yield maps from preceding years are also used as inputs with which the predictions are made in addition to other inputs. These other inputs include soil, satellite and weather data, all of them represented in spatial format in resolutions from 10 m/px up to 5 km/px.

Khanal et al. \cite{Khanal2018} performed soil variable and corn yield prediction at intra-field scale, utilizing combine harvester generated spatial yield maps as one of the target values. The authors utilized the size of the harvester head and the travelled distance of the combine harvester between each logged yield point to assign input pixels (multispectral data, various indices and DEM data) to certain yield values. Input and yield data were, thus, utilized in point-wise rather than spatial format w.r.t. modelling.

Similarly, Zhao et al. \cite{Zhao2020} utilized yield maps produced by combine harvesters as the target values for predicting wheat yields from raw and processed Sentinel-2 data. They derived various vegetation indices from multitemporal Sentinel-2 multispectral data, which were then utilized in a linear and multivariate time series model to estimate yields. While the input data was utilized as points, albeit initially spatial, the point-wise models were utilized to estimate yield maps from within-season satellite data .
