In this dissertation the applicability of novel machine learning methods with remote sensing data were studied in the context of agricultural decision support systems in smart farming. The main focus is in the utilization of high resolution unmanned aerial vehicle (UAV) data to perform in-season crop yield estimation with spatial and spatiotemporal deep learning model architectures in Finnish coastal habitat. While open-access satellite data has already been utilized in crop related modelling, such as crop type classification and yield prediction, intra-field scale prediction with smaller fields common to Nordic countries requires images with higher resolution than what is currently available from the open-access satellite systems. In addition to using UAV remote sensing data, various combinations of crop field related sensor data, data from open-access sources and satellite data are evaluated. Data quality is also an important aspect with remote sensing data, with high altitude satellite-based earth observation suffering from occasional obstructions by cloud canopy. A decision tree model is employed to estimate cloud coverage by using UAV data as cloudless groudn truth. In this disseration it is shown that with high-resolution UAV data, crop yield prediction with convolutional neural networks (CNNs) is feasible and produces results accurate enough for performing corrective farming actions in-season. Using UAV data time series not only improves the modelling performance (post-season prediction) with high-resolution UAV RGB data but also improves the predictive capabilities (in-season prediction). Furthermore, using various data sources for crop yield prediction in addition to UAV RGB data is shown to improve the predictive capabilities of the model. In summary, the use of deep learning techniques can be seen to improve the smart farming decision support pipeline by providing performant and reliable decision engines.